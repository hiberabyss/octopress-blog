\documentclass[a4paper,11pt]{article}

%可以利用 texdoc 命令查看相关包的帮助文档
\usepackage[top=1in,bottom=1in,right=1in,left=1in]{geometry}
\usepackage{amssymb}
\usepackage{amsmath}
\usepackage[colorlinks,linkcolor=blue]{hyperref}
\usepackage{booktabs}
\usepackage{xeCJK}
\usepackage{listings}

\definecolor{numColor}{gray}{0.5}
\definecolor{bodyColor}{gray}{0.9}

\lstset{
  language=c,
  breaklines=true,
  basicstyle=\footnotesize,
  backgroundcolor=\color{bodyColor},
  numbers=left,
  numberstyle=\tiny\color{numColor},
}

\setCJKmainfont[BoldFont={SimHei},ItalicFont={KaiTi}]{SimSun}
\setmainfont{Times New Roman}

\title{\bf 标题}
\author{\it 作者}
%不显示日期
\date{}

\begin{document}
\maketitle

\newcommand{\itembf}[1]{\item{\bf #1}}

\section{标题}
引用参考文献\cite{bib:test} \hfill 填充行
\subsection{子标题}
\subsubsection{子子标题}

\renewcommand\refname{参考文献}

%99用来表示最多有多少条目,用于对齐
\begin{thebibliography}{99}
    \bibitem{bib:test} 参考文献.测试
    \bibitem{bib:wikibooks} \url{http://en.wikibooks.org/wiki/LaTeX}
\end{thebibliography}
\end{document}
